\noindent \textbf{Instructions}. Answer the following multiple choice questions by selecting the correct choices.

\begin{questions} % Begins the questions environment

  % Question 1
  \question \textbf{Programming Paradigms}
  \begin{parts}
    \part Which of the following is {\it not} an example of a programming paradigm?

    \begin{oneparcheckboxes}
      \CorrectChoice JavaScript
      \choice Declarative
      \choice Imperative
      \choice Functional
      \choice Object-oriented
    \end{oneparcheckboxes}

    \part Which of the following characteristics are typical of imperative programs.

    \begin{checkboxes}
      \choice values of variables may change or ``mutate'' (they are \textit{mutable})
      \choice program execution proceeds by carrying out a sequence of instructions
      \choice functions often have \textit{side-effects}
      \CorrectChoice all of the above
    \end{checkboxes}

    \part Which of the following characteristics are typical of functional programs.
    \begin{checkboxes}
      \choice values of variables do not change or ``mutate'' (they are \textit{immutable})
      \choice functions are \textit{referentially transparent}
      \choice functions do not have \textit{side-effects}
      \CorrectChoice all of the above
    \end{checkboxes}
  \end{parts}

\vskip1cm

  % Question 2
  \question A \textit{higher-order function} is a function that
  \begin{checkboxes}
    \choice can be passed as an argument to other functions
    \choice can be returned as output by other functions
    \choice can be called a higher order of times than ordinary, ``lower-order'' functions
    \CorrectChoice accepts a function (or functions) as input or returns a function (or functions) as output.
    \choice takes a higher order of magnitude of time to return a value than ordinary, ``lower-order'' functions
  \end{checkboxes}


\vskip1cm

  % Question 3
  \question An expression \texttt{e} is called \textit{referentially transparent} provided
  \begin{checkboxes}
    \choice the value of \texttt{e}, when it is reduced to ``normal form,'' is obvious or ``transparent.''
    \choice the values all expressions to which \texttt{e} refers are obvious or ``transparent.''
    \CorrectChoice for all programs \texttt{p}, all occurrences of \texttt{e} in \texttt{p} can be replaced by the result of evaluating \texttt{e} without affecting the meaning of \texttt{p}.
    \choice none of the above
  \end{checkboxes}

\newpage

  % Question 4
  \question \textbf{Introduction to Scala, Part I}

  \begin{parts}
    \part The programming paradigm(s) of Scala is(are) which of these? (select all that apply).

    \begin{oneparcheckboxes}
      \choice assembly
      \choice declarative
      \choice imperative
      \CorrectChoice functional
      \CorrectChoice object-oriented
    \end{oneparcheckboxes}

    \part What is the result of the following program?

    \begin{verbatim}
    val x = 0
    def f(y: Int) = y + 1
    val result = {
      val x = f(3)
      x * x
    } + x
    \end{verbatim}

    \begin{oneparcheckboxes}
      \choice 0
      \CorrectChoice 16
      \choice  32
      \choice it does not terminate
    \end{oneparcheckboxes}

    \part Why should we care about writing functions that are ``tail-recursive?''

    \begin{checkboxes}
      \choice Recursion should be carried out on the tail, not the head.
      \choice Recursion should be carried out on the head, not the tail.
      \CorrectChoice Non-tail-recursive functions may exhaust stack memory.
      \choice Non-tail-recursive functions may exhaust heap memory.
    \end{checkboxes}

  \end{parts}

\vskip1cm

  % Question 5
  \question Consider the following code.

  \begin{verbatim}
    def sq(x: Double): Option[Double] =
      if (x < 0) None
      else Some(Math.sqrt(x))

    val list = List(-1.0, 4.0, 9.0)
  \end{verbatim}

  \begin{parts}
    \part To what does the expression \texttt{list.map(sq)} evaluate?

    \begin{checkboxes}
      \choice \texttt{List(2.0, 3.0)}
      \CorrectChoice \texttt{List(None, Some(2.0), Some(3.0))}
      \choice \texttt{Some(List(2.0, 3.0))}
      \choice \texttt{None}
      \choice none of the above
    \end{checkboxes}


\part To what does the expression \texttt{list.flatMap(sq)} evaluate?

    \begin{checkboxes}
      \CorrectChoice \texttt{List(2.0, 3.0)}
      \choice \texttt{List(None, Some(2.0), Some(3.0))}
      \choice \texttt{Some(List(i, 2.0, 3.0))}
      \choice \texttt{None}
      \choice none of the above
    \end{checkboxes}
  \end{parts}

  \newpage

  % Question 6
  \question \textbf{Introduction to Scala, Part II}. The parts below refer to the function \texttt{test(x:Int, y:Int) = x * x}.

  \begin{parts}
    \part For the function call \texttt{test(2, 3)}, which evaluation strategy is  most efficient (takes the least number of steps)?

    \begin{checkboxes}
      \choice call-by-value is more efficient
      \choice call-by-name is more efficient
      \CorrectChoice call-by-value and call-by-name require the same number of steps
      \choice the program does not terminate
    \end{checkboxes}

    \explain{In both cases we have to do one multiplication (\texttt{2 * 2}).}

    \part For the function call \texttt{test(3 + 4, 8)}, which evaluation strategy is most efficient?

    \begin{checkboxes}
      \CorrectChoice call-by-value is more efficient
      \choice call-by-name is  more efficient
      \choice call-by-value and call-by-name require the same number of steps
      \choice the program does not terminate
    \end{checkboxes}

    \explain{Call-by-value performs one addition (\texttt{3 + 4}) and one multiplication (\texttt{7 * 7}), whereas
      call-by-name performs two additions and one multiplication (\texttt{(3 + 4) * (3 + 4)}).}

    \part For the function call \texttt{test(7, 2*4)}, which evaluation strategy is most efficient?

    \begin{checkboxes}
      \choice call-by-value is more efficient
      \CorrectChoice call-by-name is more efficient
      \choice call-by-value and call-by-name require the same number of steps
      \choice the program does not terminate
    \end{checkboxes}

    \explain{Call-by-value performs two multiplications (\texttt{2 * 4} and \texttt{7 * 7}), whereas
      call-by-name performs just one multiplication (\texttt{7 * 7}).}

    \part For the function call \texttt{test(3+4, 2*4)} which evaluation strategy is most efficient?

    \begin{checkboxes}
      \choice call-by-value is more efficient
      \choice call-by-name is more efficient
      \CorrectChoice call-by-value and call-by-name require the same number of steps
      \choice the program does not terminate
    \end{checkboxes}

    \explain{Call-by-value performs one addition (\texttt{3 + 4}) and two multiplications (\texttt{2 * 4} and \texttt{7 * 7}), and
      call-by-name performs two additions and one multiplication (\texttt{(3 + 4) * (3 + 4)}).}

  \end{parts}
\end{questions}
