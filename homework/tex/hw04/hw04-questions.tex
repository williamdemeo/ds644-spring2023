\noindent \textbf{Instructions}. Answer the following multiple choice questions by selecting all correct choices.

\begin{questions}

  % QUESTION 1
\question[4] \textbf{Shuffling}.

\begin{parts}
  \part What is \textit{shuffling}?

  \begin{checkboxes}
  \choice a method for recovering data after hardware failure
  \choice the method used to ensure a random number generator is unbiased
  \choice any movement of data
  \choice moving data from memory to disk, usually caused by insufficient fast memory
  \CorrectChoice transferring data between nodes in a cluster, usually in order to complete a computation
  \end{checkboxes}

  \part How can shuffling sometimes be reduced or avoided using Spark?

  \begin{checkboxes}
  \choice use higher quality, fault-tolerant hardware
  \choice use a pre-shuffled random number generator
  \choice avoid algorithms that process the entire data set in favor of algorithms that only need a small subset of it
  \choice use only fast memory, eliminating all spinning disks from the network
  \CorrectChoice partition an \texttt{RDD} before applying transformations or actions that cause shuffling
  \end{checkboxes}

\end{parts}

% QUESTION 2
\question \textbf{Partitions and Partitioning}

\begin{parts}  % QUESTION 1.1

  \part Given a pair RDD (of key-value pairs), when we group values with the same key Spark collects key-value pairs with the same key on the same machine of our cluster.

  \smallskip

  \begin{oneparcheckboxes}
    \CorrectChoice True  \choice False
  \end{oneparcheckboxes}

  \medskip

  \part By default, Spark uses range partitioning to determine which key-value pair should be sent to which machine.

  \smallskip

  \begin{oneparcheckboxes}
    \choice True      \CorrectChoice False
  \end{oneparcheckboxes}

  \medskip

  \part Suppose we partition an RDD into a number of blocks.  From the following statements, select the two that are true.

  \smallskip

  \begin{checkboxes}
    \choice  A single block of the partition may be distributed across multiple machines in the cluster.
    \CorrectChoice  A block of the partition is assigned to at most one machine of the cluster.
    \choice  At least one block of the partition is assigned to every machine in the cluster.
    \choice  At most one block of the partition is assigned to every machine in the cluster.
    \CorrectChoice  More than one block of the partition may be assigned to the same machine in the cluster.
  \end{checkboxes}
\end{parts}

\newpage

% QUESTION 3
\question

Consider a Pair RDD, with keys [8, 23, 39, 40, 97], and suppose we want to partition these data into 4 blocks.

Using hash partitioning with the identity as \texttt{hashCode()} function (\texttt{n.hashCode() == n}), check the boxes next to the numbers assigned to the given partition block.

\medskip
\begin{parts}

    \part block 0:
    \begin{oneparcheckboxes}
      \CorrectChoice  8
      \choice  23
      \choice  39
      \CorrectChoice  40
      \choice  97
      \choice  none
    \end{oneparcheckboxes}

    \medskip

    \part block 1:
    \begin{oneparcheckboxes}
      \choice  8
      \choice  23
      \choice  39
      \choice  40
      \CorrectChoice  97
      \choice  none
    \end{oneparcheckboxes}

    \medskip

    \part block 2:
    \begin{oneparcheckboxes}
      \choice  8
      \choice  23
      \choice  39
      \choice  40
      \choice  97
      \CorrectChoice  none
    \end{oneparcheckboxes}

    \medskip

    \part block 3:
    \begin{oneparcheckboxes}
      \choice  8
      \CorrectChoice  23
      \CorrectChoice  39
      \choice  40
      \choice  97
      \choice  none
    \end{oneparcheckboxes}

  \end{parts}

  \bigskip

% QUESTION 4
\question

Consider a Pair RDD, with keys [8, 23, 39, 40, 97], and suppose we want to partition these data into 4 blocks.

Using range partitioning with ranges [0, 20], [21, 40], [41, 60], [61, 100], check the boxes next to the numbers assigned to the given partition block.

  \medskip

  \begin{parts}

    \part block 0:
    \begin{oneparcheckboxes}
      \CorrectChoice  8
      \choice  23
      \choice  39
      \choice  40
      \choice  97
      \choice  none
    \end{oneparcheckboxes}

    \medskip

    \part block 1:
    \begin{oneparcheckboxes}
      \choice  8
      \CorrectChoice  23
      \CorrectChoice  39
      \CorrectChoice  40
      \choice  97
      \choice  none
    \end{oneparcheckboxes}

    \medskip

    \part block 2:
    \begin{oneparcheckboxes}
      \choice  8
      \choice  23
      \choice  39
      \choice  40
      \choice  97
      \CorrectChoice  none
    \end{oneparcheckboxes}

    \medskip

    \part block 3:
    \begin{oneparcheckboxes}
      \choice  8
      \choice  23
      \choice  39
      \choice  40
      \CorrectChoice  97
      \choice  none
    \end{oneparcheckboxes}

  \end{parts}



\bigskip

% QUESTION 5
\question  Which strategy would result in a more balanced distribution of the data across the partition?

  \begin{oneparcheckboxes}
    \CorrectChoice hash partitioning
    \choice range partitioning
  \end{oneparcheckboxes}

  \explain{With both strategies, three out of four blocks are occupied.  However, with range partitioning, block 1 has 3 elements, which is 3 times larger than the next biggest block, whereas, with hash partitioning, blocks 0 and 3 each have 2 elements, which makes them equal in size and 2 times larger than the next biggest block (block 1).  Each block of a partition is assigned to a single machine in the cluster. If, in addition, we assume that
  \begin{itemize}
  \item no two blocks reside on the same machine, and
  \item the amount of work done by each machine is proportional to the amount of data assigned to that machine,
  \end{itemize}
  then it should be clear that, for the present example, hash partitioning will result in a more even distribution of work among machines in the cluster.}



\newpage


% QUESTION 6
\question

\begin{parts}

  \part %QUESTION 6.1

  Which method can we use to determine whether Spark recognizes that a transformation or action will result in shuffling?

  \begin{oneparcheckboxes}
    \choice \texttt{debugDAG}
    \choice \texttt{isShuffled}
    \choice \texttt{showSchema}
    \choice \texttt{showExecutionPlan}
    \CorrectChoice \texttt{toDebugString}
  \end{oneparcheckboxes}


  \part %QUESTION 6.2

  How data is initially partitioned and arranged on the cluster doesn't matter, since Spark will always re-arrange your data to avoid shuffling.

  \begin{oneparcheckboxes}
    \choice True
    \CorrectChoice False
  \end{oneparcheckboxes}


  \part %QUESTION 6.3

  \texttt{reduceByKey} running on a pre-partitioned ROD will computed values locally, requiring only the final reduced values to
  be sent from workers to the driver.

  \begin{oneparcheckboxes}
    \CorrectChoice True
    \choice False
  \end{oneparcheckboxes}

  \part %QUESTION 6.4

  \texttt{join} called on two RDDs that are pre-partitioned with the same partitioner and cached on the same node will cause
  the join to be computed locally, with no shuffling across the network.

  \begin{oneparcheckboxes}
    \CorrectChoice True
    \choice False
  \end{oneparcheckboxes}


  \part %QUESTION 6.5

  Suppose algorithm \textbf{A} joins two RDDs and then performs a filter on the result while algorithm \textbf{B} performs a filter on the two RDDs and then joins the results.  Assume the two algorithms obtain the same result.  In general, which algorithm do you expect will cause less data shuffling?

  \begin{oneparcheckboxes}
    \choice \textbf{A}
    \CorrectChoice \textbf{B}
  \end{oneparcheckboxes}


\end{parts}

\bigskip

%QUESTION 7
\question
Answer the following parts by typing in the spaces provided.  Select from among the following words or phrases: ``at most one,'' ``multiple,'' ``fast,'' ``slow,'' ``some,'' or ``none.''

\medskip
\begin{parts}

  \part

  In a \textit{narrow dependency}, each block of the parent RDD may be used by \fillin[at most one] block(s) of the child RDD.
  \medskip

  Narrow dependencies are \fillin[fast] since they require \fillin[none] of the data to be shuffled.

  \medskip
  \part

  In a \textit{wide dependency}, each block of the parent RDD may be used by \fillin[multiple] block(s) of the child RDD.

  \medskip
  Wide dependencies are \fillin[slow] since they require \fillin[some] of the data to be shuffled.

\end{parts}

\question % QUESTION 8
\begin{parts}
  \part % QUESTION 8.1
  The \textit{query optimizer} of Spark SQL is called
  \smallskip

  \begin{oneparcheckboxes}
    \CorrectChoice Catalyst
    \choice Cobalt
    \choice Map Reduce
    \choice Platinum
    \choice Tungsten
  \end{oneparcheckboxes}


  \part % QUESTION 8.2
  The \textit{off-heap serializer} of Spark SQL is called

  \smallskip
  \begin{oneparcheckboxes}
    \choice Catalyst
    \choice Cobalt
    \choice Map Reduce
    \choice Platinum
    \CorrectChoice Tungsten
  \end{oneparcheckboxes}

\end{parts}


\end{questions}
